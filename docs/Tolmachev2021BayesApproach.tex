\documentclass[12pt, twoside]{article}
\usepackage{jmlda}
\newcommand{\hdir}{.}

\begin{document}

\title
    [Байесовский выбор структур обобщенно-линейных моделей] % краткое название; не нужно, если полное название влезает в~колонтитул
    {Байесовский выбор структур обобщенно-линейных моделей}
\author
    [A.\,Д.~Толмачев] % список авторов (не более трех) для колонтитула; не нужен, если основной список влезает в колонтитул
    {A.\,Д.~Толмачев, А.\,А.~Адуенко, В.\,В.~Стрижов} % основной список авторов, выводимый в оглавление
    [А.\,Д.~Толмачев, А.\,А.~Адуенко, В.\,В.~Стрижов] % список авторов, выводимый в заголовок; не нужен, если он не отличается от основного
\email
    {tolmachev.a.d.@phystech.edu; aduenko1@gmail.com; strijov@ccas.ru}
\thanks
    {Работа выполнена в рамках курса <<Моя первая научная статья>>, НИУ МФТИ, 2021}
    
\abstract
    {В данной работе исследуется проблема мультиколлинеарности и её влияние
    на эффективность методов выбора признаков. Рассматривается задача выбора признаков и различные подходы к ее решению. Исследованы возможности применения байесовского подхода для метода отбора признаков на основе квадратичного программирования. В работе приводятся критерии сравнения различных способов отбора признаков, и проведено сравнение различных методов на тестовых выборках.
    Сделан вывод об эффективности рассматриваемых подходов на определенных типах данных.

\bigskip
\noindent
\textbf{Ключевые слова}: \emph {регрессионный анализ; мультиколлинеарность; байесовский подход; выбор признаков;  квадратичное программирование}
}

%данные поля заполняются редакцией журнала
%\doi{}
%\receivedRus{}
%\receivedEng{}

\maketitle
\linenumbers

\section{Введение}
Работа посвящена анализу применения байесовского подхода для методов отбора признаков и сравнительному анализу различных методов отбора признаков. Предполагается, что исследуемая выборка содержит значительное число мультиколлинеарных признаков. Мультиколлинеарность — это сильная корреляционная связь между отбираемыми для анализа признаками, совместно
воздействующими на целевой вектор, которая затрудняет оценивание
регрессионных параметров и выявление зависимости между признаками и целевым вектором. Проблема мультиколлинеарности и возможные способы её обнаружения и устранения описаны в \cite{SneeRon, Leamer}. 

Задача выбора оптимального подмножества признаков является одной из основных задач предварительной обработки данных. Методы выбора признаков
основаны на минимизации некоторого функционала, который отражает качество
рассматриваемого подмножества признаков. В \cite{FeatureSelection1, FeatureSelection2} сделан обзор основных существующих методов отбора признаков.

В \cite{KatrutsaS17} предложен новый метод отбора признаков, использующий один из основных методов оптимизации, квадратичное программирование, для задачи отбора признаков. Цель данной работы состоит в анализе возможностей применения байесовского подхода для метода квадратичного программирования в задаче выбора признаков. 

Важной частью этой работы является сравнение метода на основе байесовского подхода и других методов отбора признаков, описанных, например, в \cite{Katrutsa15}, на различных тестовых выборках.

\section{Название раздела}
Данный документ демонстрирует оформление статьи,
подаваемой в электронную систему подачи статей \url{http://jmlda.org/papers} для публикации в журнале <<Машинное обучение и анализ данных>>.
Более подробные инструкции по~стилевому файлу \texttt{jmlda.sty} и~использованию издательской системы \LaTeXe\
находятся в~документе \texttt{authors-guide.pdf}.
Работу над статьёй удобно начинать с~правки \TeX-файла данного документа.

Обращаем внимание, что данный документ должен быть сохранен в кодировке~\verb'UTF-8 without BOM'.
Для смены кодировки рекомендуется пользоваться текстовыми редакторами \verb'Sublime Text' или \verb'Notepad++'.

\paragraph{Название параграфа}
Разделы и~параграфы, за исключением списков литературы, нумеруются.

\section{Заключение}
Желательно, чтобы этот раздел был, причём он не~должен дословно повторять аннотацию.
Обычно здесь отмечают, каких результатов удалось добиться, какие проблемы остались открытыми.


%%%% если имеется doi цитируемого источника, необходимо его указать, см. пример в \bibitem{article}
%%%% DOI публикации, зарегистрированной в системе Crossref, можно получить по адресу http://www.crossref.org/guestquery/


\bibliographystyle{plain}
\bibliography{Tolmachev2021BayesApproach.bib}

\end{document}
