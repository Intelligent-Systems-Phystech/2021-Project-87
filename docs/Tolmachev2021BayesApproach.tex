\documentclass[12pt, twoside]{article}
\usepackage{jmlda}
\newcommand{\hdir}{.}

% notations
% bold
\newcommand{\ba}{\mathbf{a}}
\newcommand{\bz}{\mathbf{z}}
\newcommand{\bx}{\mathbf{x}}
\newcommand{\by}{\mathbf{y}}
\newcommand{\bw}{\mathbf{w}}
\newcommand{\bfx}{\mathbf{f}}
\newcommand{\bb}{\mathbf{b}}
\newcommand{\bu}{\mathbf{u}}
\newcommand{\bX}{\mathbf{X}}
\newcommand{\bZ}{\mathbf{Z}}
\newcommand{\bA}{\mathbf{A}}
\newcommand{\bB}{\mathbb{B}}
\newcommand{\bI}{\mathbf{I}}
\newcommand{\bJ}{\mathcal{J}}
\newcommand{\bV}{\mathbf{V}}
\newcommand{\bU}{\mathbf{U}}
\newcommand{\bG}{\mathbf{G}}
\newcommand{\bQ}{\mathbf{Q}}
\newcommand{\btheta}{\boldsymbol{\theta}}
\newcommand{\bPsi}{\boldsymbol{\Psi}}
\newcommand{\bpsi}{\boldsymbol{\psi}}
\newcommand{\bxi}{\boldsymbol{\xi}}
\newcommand{\bchi}{\boldsymbol{\chi}}
\newcommand{\bzeta}{\boldsymbol{\zeta}}
\newcommand{\blambda}{\boldsymbol{\lambda}}
\newcommand{\beps}{\boldsymbol{\varepsilon}}
\newcommand{\bZeta}{\boldsymbol{Z}}
% mathcal
\newcommand{\cX}{\mathcal{X}}
\newcommand{\cY}{\mathcal{Y}}
\newcommand{\cW}{\mathcal{W}}
% transpose
\newcommand{\getT}{^{\mathsf{T}}}


\begin{document}

\title
    [Байесовский выбор структур обобщенно-линейных моделей] % краткое название; не нужно, если полное название влезает в~колонтитул
    {Байесовский выбор структур обобщенно-линейных моделей}
\author
    [A.\,Д.~Толмачев] % список авторов (не более трех) для колонтитула; не нужен, если основной список влезает в колонтитул
    {A.\,Д.~Толмачев, А.\,А.~Адуенко, В.\,В.~Стрижов} % основной список авторов, выводимый в оглавление
    [А.\,Д.~Толмачев, А.\,А.~Адуенко, В.\,В.~Стрижов] % список авторов, выводимый в заголовок; не нужен, если он не отличается от основного
\email
    {tolmachev.a.d.@phystech.edu; aduenko1@gmail.com; strijov@ccas.ru}
\thanks
    {Работа выполнена в рамках курса <<Моя первая научная статья>>, НИУ МФТИ, 2021}
    
\abstract
    {В данной работе исследуется проблема мультиколлинеарности и её влияние
    на эффективность методов выбора признаков. Рассматривается задача выбора признаков и различные подходы к ее решению. Исследованы возможности применения байесовского подхода для метода отбора признаков на основе квадратичного программирования. В работе приводятся критерии сравнения различных способов отбора признаков, и проведено сравнение различных методов на тестовых выборках.
    Сделан вывод об эффективности рассматриваемых подходов на определенных типах данных.

\bigskip
\noindent
\textbf{Ключевые слова}: \emph {регрессионный анализ; мультиколлинеарность; байесовский подход; выбор признаков;  квадратичное программирование}
}

%данные поля заполняются редакцией журнала
%\doi{}
%\receivedRus{}
%\receivedEng{}

\maketitle
\linenumbers

\section{Введение}
Работа посвящена анализу применения байесовского подхода для методов отбора признаков и сравнительному анализу различных методов отбора признаков. Предполагается, что исследуемая выборка содержит значительное число мультиколлинеарных признаков. Мультиколлинеарность — это сильная корреляционная связь между отбираемыми для анализа признаками, совместно
воздействующими на целевой вектор, которая затрудняет оценивание
регрессионных параметров и выявление зависимости между признаками и целевым вектором. Проблема мультиколлинеарности и возможные способы её обнаружения и устранения описаны в \cite{SneeRon, Leamer}. 

Задача выбора оптимального подмножества признаков является одной из основных задач предварительной обработки данных. Методы выбора признаков
основаны на минимизации некоторого функционала, который отражает качество
рассматриваемого подмножества признаков. В \cite{FeatureSelection1, FeatureSelection2} сделан обзор основных существующих методов отбора признаков.

В \cite{KatrutsaS17} предложен новый метод отбора признаков, использующий один из основных методов оптимизации, квадратичное программирование, для задачи отбора признаков. Цель данной работы состоит в анализе возможностей применения байесовского подхода для метода квадратичного программирования в задаче выбора признаков. 

Важной частью этой работы является сравнение метода на основе байесовского подхода и других методов отбора признаков, описанных, например, в \cite{Katrutsa15}, на различных тестовых выборках.


\section{Постановка задачи}

\subsection{Применение квадратичной оптимизации для задачи отбора признаков}

В \cite{KatrutsaS17} предлагается подход с применением квадратичной оптимизации для задачи выбора признаков. Основная идея предлагаемого подхода заключается в минимизации количества схожих признаков и максимихации количества релевантных признаков.
Пусть $\bJ$ -- множество признаков в рассматриваемой модели, и $|\bJ| = n$. 
Зададим функционал $Q(\ba) = \ba \getT \bQ \ba - \bb \getT \ba$, где $\ba \in \mathbb{R}^n$ и $\bQ \in \mathbb{R}^{n \times n}$ -- матрица схожести признаков, а $\bb \in \mathbb{R}^n$ -- вектор релевантности признаков с целевым вектором.
Матрицу $\bQ$ и вектор $\bb$ будем представлять как функции Sim и Rel соответственно, где Sim: $\bJ \times \bJ \rightarrow [0, 1]$, Rel: $\bJ \rightarrow [0, 1]$. Таким образом, необходимо решить задачу оптимизации: 

$$ \ba^* = \arg \min_{a \in \bB^n} Q(\ba).$$

Важно отметить, что задача целочисленного квадратичного программирования, сформулированная выше, является $\mathbf{NP}$-полной, так как поиск минимума функции $\bQ$ ведется по вершинам булева куба $\bB^n = \{0, 1\}^n$. Поэтому, чтобы можно было применять различные методы выпуклой оптимизации будем искать минимум функции по выпуклой оболочке булева куба Conv$(\bB^n) = [0, 1]^n$. 

Тогда получаем следующую задачу выпуклой оптимизации:
\begin{equation*}
\begin{cases}
   $$ \bz^* = \arg \min_{z \in [0, 1]^n}  \bz \getT \bQ \bz - \bb \getT \bz$$  \\
   $$\|z\|_1 \le 1$$
 \end{cases}
 \end{equation*}

Здесь, $Q$ и $b$ по-прежнему являются матрицей схожести признаков и вектором релевантности признаков соответственно. В данной работе функции Sym и Rel (или, другими словами, матрица $Q$ и вектор $b$ в обозначениях выше) задаются заранее до применения метода на основе сходств между признаками в датасете.

Далее положим, $\tau$ - пороговое значение для отбора признаков в данном методе, т.е. $z_i^* > \tau \Leftrightarrow a_i^* = 1 \Leftrightarrow j \in \mathcal{A}$, где $\mathcal{A} \subset \bJ$ -- множество отобранных методом признаков.

Далее предлагается рассмотреть возможные применения байесовского подхода к данному методу квадратичного программирования.


\subsection{Данные для экспериментов}

В качестве данных для экспериментов по проверке предложенных подходов мы используем синтетические наборы данных из работы \cite{Katrutsa15}, в которых рассматриваются различные типы зависимости признаков между собой и с целевым вектором. Кроме того, будут проведены эксперименты и на ссобственных генерированных синтетических данных.


\section{Базовый эксперимент}

\subsection{Цель}

Как сказано в \cite{KatrutsaS17} метод квадратичного программирования улучшает качество отбора признаков для многих типов выборок. Однако, не во всех случаях этот метод дает наилучшие результаты. Цель базового эксперимента заключается в поиске и рассмотрении выборок с мультиколлинеарными признаками, на которых методу квадратичного программирования не удается провести качественный отбор признаков. Далее планируется исследовать, в чем сходство выборок, на которых метод квадратичного программирования дает неоптимальные результаты, чтобы учесть полученные закономерности при разработке нового метода отбора признаков на основе байесовского подхода.

\subsection{Описание данных}

В качестве базового эксперимента рассмотрим синтетические данные и применим на них метод квадратичного программирования QBFS, описанный выше, для поиска значения $\ba^*$, при котором значение функционала $Q(\ba)$ принимает наименьшее значение.

\subsection{Результаты эксперимента}

Результаты эксперимента и базовый код будут загружены немного позже. Надеюсь на понимание.





\section{Название раздела}
Данный документ демонстрирует оформление статьи,
подаваемой в электронную систему подачи статей \url{http://jmlda.org/papers} для публикации в журнале <<Машинное обучение и анализ данных>>.
Более подробные инструкции по~стилевому файлу \texttt{jmlda.sty} и~использованию издательской системы \LaTeXe\
находятся в~документе \texttt{authors-guide.pdf}.
Работу над статьёй удобно начинать с~правки \TeX-файла данного документа.

Обращаем внимание, что данный документ должен быть сохранен в кодировке~\verb'UTF-8 without BOM'.
Для смены кодировки рекомендуется пользоваться текстовыми редакторами \verb'Sublime Text' или \verb'Notepad++'.

\paragraph{Название параграфа}
Разделы и~параграфы, за исключением списков литературы, нумеруются.

\section{Заключение}
Желательно, чтобы этот раздел был, причём он не~должен дословно повторять аннотацию.
Обычно здесь отмечают, каких результатов удалось добиться, какие проблемы остались открытыми.


\bibliographystyle{plain}
\bibliography{Tolmachev2021BayesApproach.bib}

\end{document}
